
\documentclass[12pt]{article}
\usepackage{listings}
\usepackage{pdfpages}
\usepackage{amsmath}
\usepackage[utf8]{inputenc}
\usepackage[english]{babel}
\usepackage{multicol}
\usepackage{babel}

\usepackage[margin=1in]{geometry}
\begin{document}
\begin{titlepage}
   \begin{center}
       \vspace*{1cm}
       \Large
       AoA Programming Assignment 3
       \normalsize

       \vspace{0.5cm}

       Author: Gabriel Hofer

       \vspace{0.5cm}

       Course: CSC-372

       \vspace{0.5cm}

       Instructor: Dr. Rebenitsch
       \vspace{0.5cm}

       Due: November 5, 2020

       \vfill

       Computer Science and Engineering\

       South Dakota School of Mines and Technology\
   \end{center}
\end{titlepage}
%------------------------------------------------------------------------------------
\newpage
%\footnotesize
\small
\section*{Question 1}
\indent 
\par
The recursive calls of the closest points function form a binary tree because the set of points is 
divided in half and two recursive calls are made: one for the right half and one for the left. 
After those calls have returned, we know the closest pair of points in each half of the set. 

So, in order to make a dendogram, we simply need to show the structure of the recursive calls by 
creating a single node for the current recursive call. Then we add two edges connecting each half 
of the set to the current node or function call. Also, note that the base case probably connects two
actual points to each other, although this is slightly implementation dependent.

\begin{verbatim}
1. Sort the points from smallest to largest x coordinate (X)
2. Sort the points from smallest to largest y coordinate (Y)
3. Evenly divide the set of points into "left" and "right" sets (both X and Y!)
4. Repeat division recursively until there are only 2 points
5. To merge, the closest pair of points is either the closest pair 
   returned from both sides or is across the split boundary as follows:
  a. find all the Y points within d of the split line where d is 
     the distance between the closest points from both halves.
  b. check the nearby points that could be closer (there are only 7 max)
  c. update the minimum pair 
\end{verbatim}

\section*{Question 2}
\textbf{a. [5 points] What is the optimal substructure (you must name the specific item being removed)?}

The subproblems in this problem occur when we add some subset of rods in the total set of rods that we have available $SomeRods \subset AllRods$ to the beam. Also, in our algorithm, we always add the smallest rod to the beam. Thus, $SomeRodswill$ contain the first $SomeRods.size$ smallest rods from the set of all rods $AllRods$. 

\noindent \textbf{b. [6 points] What is the greedy property? }

Greedy Choice Property: A global optimum can be attained by selecting a local optimum. For this problem, every time we add a rod to the beam we maintain that the current solution is optimal for the subset of rods that have already been added to the beam AND this locally optimal solution will be used to build more globally optimal solutions. 

\newpage
\noindent \textbf{c. [9 points] Prove correctness of your algorithm. A proof of contradiction is acceptable.}
\lstset{
  basicstyle = {\footnotesize\ttfamily},
  xleftmargin=0in,
}
\begin{lstlisting}[frame=single]
MAKE-BEAM:
  layer:=1                                                        
  while remainingRods.size() > 0  do 
    beam[ layer ].add(min(remainingRods))   // add shortest rod to beam
    remainingRods.erase(min(remainingRods)) // remove rod from list
    rodCnt[ layer ] += 1                    // count rods in this layer
    if remainingRods.size() <= rodCnt[ layer ] then
      rodCnt[ layer ] += remainingRods.size()
      break while loop 
    inc layer                               // goto next layer
\end{lstlisting}

\textbf{Proof: } While the length of the current layer is less than or
equal to the length of the previous layer, add the shortest rod that hasn't 
been welded to a beam to the current layer. 
Every time we add a rod to the beam we also have to check whether there 
are not enough rods to make another layer after the current layer is completed. 
In order words, if $ remainingRods.size() \le rodCnt[ layer ] $
(i.e. if the number of rods that haven't been welded to the beam is less than 
or equal to the number of rods in the current layer), 
then we simply add all of the 
remaining rods to the current layer. We do this because otherwise, we would 
violate the criterion that each layer must have more layers than the 
previous layer. 
This will happen every time the number of rods in the set is not a triangle 
number.


\section*{Question 3}

\begin{center}
  \begin{tabular}{ | l | l | l | l | l | }
    \hline
    Step & Add/Remove & Segments   & Lines Compared & Intersection? \\ \hline
    1    & Add A       & A          & NA    & N \\ \hline
    2    & Add B       & B, A       & -     & N \\ \hline
    3    & Add C       & B, A, C    & -     & N \\ \hline
    4    & Add D       & B, D, A, C & -     & N \\ \hline
    5    & Remove A    & B, D, C    & -     & N \\ \hline
    6    & Remove B    & D, C       & -     & N \\ \hline
    7    & Add E       & D, C, E    & -     & N \\ \hline
    8    & Remove C    & D, E       & -     & N \\ \hline
    9    & Remove D    & E          & -     & N \\ \hline
  \end{tabular}
\end{center}





\end{document}


